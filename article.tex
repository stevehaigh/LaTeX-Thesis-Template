% Options for packages loaded elsewhere
% Options for packages loaded elsewhere
\PassOptionsToPackage{unicode}{hyperref}
\PassOptionsToPackage{hyphens}{url}
\PassOptionsToPackage{dvipsnames,svgnames,x11names}{xcolor}
%
\documentclass[
  9pt,
  letterpaper,
  twocolumn]{article}
\usepackage{xcolor}
\usepackage[top=0.75in,bottom=0.75in,left=0.75in,right=0.75in]{geometry}
\usepackage{amsmath,amssymb}
\setcounter{secnumdepth}{-\maxdimen} % remove section numbering
\usepackage{iftex}
\ifPDFTeX
  \usepackage[T1]{fontenc}
  \usepackage[utf8]{inputenc}
  \usepackage{textcomp} % provide euro and other symbols
\else % if luatex or xetex
  \usepackage{unicode-math} % this also loads fontspec
  \defaultfontfeatures{Scale=MatchLowercase}
  \defaultfontfeatures[\rmfamily]{Ligatures=TeX,Scale=1}
\fi
\usepackage{lmodern}
\ifPDFTeX\else
  % xetex/luatex font selection
\fi
% Use upquote if available, for straight quotes in verbatim environments
\IfFileExists{upquote.sty}{\usepackage{upquote}}{}
\IfFileExists{microtype.sty}{% use microtype if available
  \usepackage[]{microtype}
  \UseMicrotypeSet[protrusion]{basicmath} % disable protrusion for tt fonts
}{}
\usepackage{setspace}
\makeatletter
\@ifundefined{KOMAClassName}{% if non-KOMA class
  \IfFileExists{parskip.sty}{%
    \usepackage{parskip}
  }{% else
    \setlength{\parindent}{0pt}
    \setlength{\parskip}{6pt plus 2pt minus 1pt}}
}{% if KOMA class
  \KOMAoptions{parskip=half}}
\makeatother
% Make \paragraph and \subparagraph free-standing
\makeatletter
\ifx\paragraph\undefined\else
  \let\oldparagraph\paragraph
  \renewcommand{\paragraph}{
    \@ifstar
      \xxxParagraphStar
      \xxxParagraphNoStar
  }
  \newcommand{\xxxParagraphStar}[1]{\oldparagraph*{#1}\mbox{}}
  \newcommand{\xxxParagraphNoStar}[1]{\oldparagraph{#1}\mbox{}}
\fi
\ifx\subparagraph\undefined\else
  \let\oldsubparagraph\subparagraph
  \renewcommand{\subparagraph}{
    \@ifstar
      \xxxSubParagraphStar
      \xxxSubParagraphNoStar
  }
  \newcommand{\xxxSubParagraphStar}[1]{\oldsubparagraph*{#1}\mbox{}}
  \newcommand{\xxxSubParagraphNoStar}[1]{\oldsubparagraph{#1}\mbox{}}
\fi
\makeatother


\usepackage{longtable,booktabs,array}
\usepackage{calc} % for calculating minipage widths
% Correct order of tables after \paragraph or \subparagraph
\usepackage{etoolbox}
\makeatletter
\patchcmd\longtable{\par}{\if@noskipsec\mbox{}\fi\par}{}{}
\makeatother
% Allow footnotes in longtable head/foot
\IfFileExists{footnotehyper.sty}{\usepackage{footnotehyper}}{\usepackage{footnote}}
\makesavenoteenv{longtable}
\usepackage{graphicx}
\makeatletter
\newsavebox\pandoc@box
\newcommand*\pandocbounded[1]{% scales image to fit in text height/width
  \sbox\pandoc@box{#1}%
  \Gscale@div\@tempa{\textheight}{\dimexpr\ht\pandoc@box+\dp\pandoc@box\relax}%
  \Gscale@div\@tempb{\linewidth}{\wd\pandoc@box}%
  \ifdim\@tempb\p@<\@tempa\p@\let\@tempa\@tempb\fi% select the smaller of both
  \ifdim\@tempa\p@<\p@\scalebox{\@tempa}{\usebox\pandoc@box}%
  \else\usebox{\pandoc@box}%
  \fi%
}
% Set default figure placement to htbp
\def\fps@figure{htbp}
\makeatother





\setlength{\emergencystretch}{3em} % prevent overfull lines

\providecommand{\tightlist}{%
  \setlength{\itemsep}{0pt}\setlength{\parskip}{0pt}}



 
\usepackage[]{biblatex}
\addbibresource{bibliography/references.bib}


\usepackage{times}
\usepackage{authblk}
\makeatletter
\@ifpackageloaded{caption}{}{\usepackage{caption}}
\AtBeginDocument{%
\ifdefined\contentsname
  \renewcommand*\contentsname{Table of contents}
\else
  \newcommand\contentsname{Table of contents}
\fi
\ifdefined\listfigurename
  \renewcommand*\listfigurename{List of Figures}
\else
  \newcommand\listfigurename{List of Figures}
\fi
\ifdefined\listtablename
  \renewcommand*\listtablename{List of Tables}
\else
  \newcommand\listtablename{List of Tables}
\fi
\ifdefined\figurename
  \renewcommand*\figurename{Figure}
\else
  \newcommand\figurename{Figure}
\fi
\ifdefined\tablename
  \renewcommand*\tablename{Table}
\else
  \newcommand\tablename{Table}
\fi
}
\@ifpackageloaded{float}{}{\usepackage{float}}
\floatstyle{ruled}
\@ifundefined{c@chapter}{\newfloat{codelisting}{h}{lop}}{\newfloat{codelisting}{h}{lop}[chapter]}
\floatname{codelisting}{Listing}
\newcommand*\listoflistings{\listof{codelisting}{List of Listings}}
\makeatother
\makeatletter
\makeatother
\makeatletter
\@ifpackageloaded{caption}{}{\usepackage{caption}}
\@ifpackageloaded{subcaption}{}{\usepackage{subcaption}}
\makeatother
\usepackage{bookmark}
\IfFileExists{xurl.sty}{\usepackage{xurl}}{} % add URL line breaks if available
\urlstyle{same}
\hypersetup{
  pdftitle={Title of Your PNAS Article},
  pdfauthor={Author One; Author Two; Author Three},
  pdfkeywords={Keyword 1, Keyword 2, Keyword 3, Keyword 4},
  colorlinks=true,
  linkcolor={blue},
  filecolor={Maroon},
  citecolor={Blue},
  urlcolor={Blue},
  pdfcreator={LaTeX via pandoc}}


\title{Title of Your PNAS Article}
\author{Author One \and Author Two \and Author Three}
\date{2026-01-22}
\begin{document}
\maketitle


\setstretch{1}
\subsection*{Abstract}\label{abstract}
\addcontentsline{toc}{subsection}{Abstract}

Your abstract goes here (maximum 250 words). Provide a clear and concise
summary of the study's purpose, methods, key findings, and significance.
The abstract should be understandable to a broad scientific audience and
must be written in a single paragraph. References in the abstract must
be cited in full within the abstract itself.

\subsection*{Significance Statement}\label{significance-statement}
\addcontentsline{toc}{subsection}{Significance Statement}

Your significance statement goes here (50-120 words). Explain the
relevance and importance of your work in broad context to a readership
outside your specific field. Focus on why this research matters and what
new insights it provides. Write at a level understandable to an
undergraduate-educated scientist.

\subsection*{Introduction}\label{introduction}
\addcontentsline{toc}{subsection}{Introduction}

\textbf{Do not} include the word ``Introduction'' as a section heading
(PNAS style). Start your introduction text here without a heading.

Provide background and context for your research. Establish the problem
or gap in knowledge that your study addresses. Keep this section
concise---PNAS articles emphasize brevity.

Cite references like this \autocite{smith2020} or
\autocite{jones2019,brown2021}.

\subsection{Results}\label{results}

Present your main findings in a logical sequence. Use subsections as
needed to organize the content. PNAS encourages results-first
presentation.

\subsubsection{Subsection 1}\label{subsection-1}

Results text here. Reference figures like Figure~\ref{fig-example1} and
tables like \textbf{?@tbl-example1}.

\begin{figure}

\centering{

\includegraphics[width=0.8\linewidth,height=\textheight,keepaspectratio]{figures/placeholder.png}

}

\caption{\label{fig-example1}Example figure caption. Describe what is
shown clearly and concisely.}

\end{figure}%

\subsubsection{Subsection 2}\label{subsection-2}

Continue with additional results. For complex tables that need to span
columns, use LaTeX directly or place in Supporting Information.

\subsection{Discussion}\label{discussion}

Interpret your results in the context of existing knowledge. Address the
significance and implications of your findings. Discuss limitations and
future directions. Keep this section focused and avoid excessive
speculation.

\subsection*{Materials and Methods}\label{materials-and-methods}
\addcontentsline{toc}{subsection}{Materials and Methods}

Provide sufficient methodological detail to allow others to reproduce
your work. PNAS places methods at the end of the article. If methods are
particularly extensive, consider moving detailed protocols to the
Supporting Information.

\subsubsection{Study Design}\label{study-design}

Describe your experimental or computational approach.

\subsubsection{Data Collection}\label{data-collection}

Detail how data were collected, including sample sizes, equipment, and
protocols.

\subsubsection{Statistical Analysis}\label{statistical-analysis}

Describe statistical methods and software used. Report significance
thresholds (e.g., P \textless{} 0.05).

\subsubsection{Ethics Statement}\label{ethics-statement}

If applicable, identify the institutional review board (IRB) or ethics
committee that approved the study. Describe informed consent procedures
for human participants or animal care protocols.

\subsection*{Data Availability}\label{data-availability}
\addcontentsline{toc}{subsection}{Data Availability}

State how readers can access your data, code, and materials. PNAS
requires data sharing in public repositories. Example: ``All data and
analysis code are available at {[}repository URL{]}. Raw sequencing data
have been deposited in the NCBI Sequence Read Archive under accession
number XXXXX.''

\subsection*{Acknowledgments}\label{acknowledgments}
\addcontentsline{toc}{subsection}{Acknowledgments}

Acknowledge funding sources, technical assistance, and other
contributions. Do not include acknowledgments in the Supporting
Information.

Example: ``This work was supported by National Institutes of Health
Grant R01-XXXXXX. We thank {[}Name{]} for technical assistance.''

\subsection*{References}\label{references}
\addcontentsline{toc}{subsection}{References}

\printbibliography[heading=none]

\begin{center}\rule{0.5\linewidth}{0.5pt}\end{center}

\subsection{Supporting Information}\label{supporting-information}

\textbf{Note}: Supporting Information (SI) should typically be a
separate document (\texttt{si-appendix.qmd}). This section is included
here as a placeholder to show the structure.

\subsubsection{SI Materials and Methods}\label{si-materials-and-methods}

Extended methodological details that are essential but too lengthy for
the main text.

\subsubsection{SI Figures}\label{si-figures}

Supplementary figures would go here (create \texttt{si-appendix.qmd} for
actual SI content).

\subsubsection{SI Tables}\label{si-tables}

Supplementary tables would go here (create \texttt{si-appendix.qmd} for
actual SI content).

\subsubsection{SI References}\label{si-references}

Additional references cited only in the SI.





\end{document}
